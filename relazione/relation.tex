\documentclass[a4paper,12pt]{report}
\usepackage[italian]{babel}
\usepackage[italian]{cleveref}
\usepackage{fancyhdr, graphicx}


\title{Progetto di \\"Programmazione di Reti"\\Chat Game}
\author{Davide Merli, Ryan Perrina }
\date{\today}

\begin{document}

\maketitle
\tableofcontents


\chapter{Descrizione generale}

Il software, realizzato come elaborato del corso di “Programmazione di Reti” del corso di
laurea in Ingegneria e Scienze Informatiche dell’università di Bologna, anno accademico 2020-2021,
si pone come obiettivo la realizzazione di una implementazione della traccia n. 3 proposta dal
professore Andrea Pirrodi:

"Sfruttando il principio della CHAT vista a lezione implementate un’architettura
client-server per il supporto di un Multiplayer Playing Game testuale.
I giocatori che accedono alla stanza sono accolti dal Master (server) che assegna
loro un ruolo e propone loro un menu con tre opzioni, due delle quali
celano una domanda mentre la terza è l’opzione trabocchetto. Se sceglie
l’opzione trabocchetto si viene eliminati dal gioco e quindi si esce dalla chat.
Se si seleziona invece una delle domande e  si risponde correttamente al quesito si
acquisisce un punto, in caso contrario si perde un punto.
Il gioco ha una durata temporale finita; il giocatore che al termine del tempo
ha acquisito più punti è il vincitore."

\section{Istruzioni di gioco}
Innanzitutto, il server deve essere eseguito, in modo da aprire le connessioni ai client.
Quando un client vuole partecipare al gioco, gli viene mostrata una schermata contenente
un'area per lo scambio di messaggi e tre pulsanti: "Invia", "Pronto" ed "Esci".
La prima cosa richiesta al client è di inviare un messaggio contenente il suo nome: se il nome
è già presente verrà chiesto di reinserire un nome diverso, altrimenti il client viene ammesso
e potrà partecipare alla partita.
Successivamente, al client viene assegnato uno di sei ruoli disponibili, i quali hanno effetti sul
punteggio ottenuto dalle domande.
Quando un giocatore è entrato, può cliccare sul pulsante "Pronto"; quando tutti i giocatori presenti
hanno dichiarato di essere pronti il gioco comincia, e non potranno entrare in gioco altri client.

All'inizio del gioco appare una seconda area di testo in cui verranno visualizzate le domande, seguita
dal pulsante "Rispondi" e da tre pulsanti etichettati con "A", "B" e "C". Per ricevere una domanda
bisogna cliccare su uno di questi tre pulsanti, ma uno dei tre contiene un tranello: premendo tale
pulsante, la finestra di gioco verrà chiusa, quindi il giocatore avrà perso la partita.
Per rispondere alla domanda bisogna scrivere la risposta in un apposito spazio e cliccare sul pulsante "Rispondi":
a quel punto il giocatore verrà informato se ha risposto in modo corretto o errato, e verrà visualizzato
il suo punteggio complessivo.
All'inzio del gioco, inoltre, viene avviato un timer di durata 2 minuti: allo scadere del tempo il giocatore
con più punti è il vincitore.

\section{Domande}
Ogni domanda rientra sotto una delle sei materie scelte (scienze, geografia, storia, sport, spettacolo, arte):
tali materie sono ispirate a quelle del gioco da tavolo "Trivial Pursuit", prodotto dalla Hasbro e dalla
Horn Abbot. Le domande stesse, ad onor del vero, sono state selezionate dall'edizione del 2016 di Trivial Pursuit.

\section{Ruoli dei giocatori}
Normalmente, quando un giocatore risponde ad una domanda, gli viene aggiunto un punto se la risposta
è corretta, o gliene viene sottratto uno se la risposta è sbagliata. Lo scopo dei ruoli è di rendere
questa meccanica più "interessante", facendo sì che alcune domande abbiano un valore diverso a seconda
del giocatore che vi risponde, e del ruolo assegnatogli. In seguito una veloce spiegazione dei ruoli:
\begin{itemize}
	\item Scienziato: guadagna due punti se risponde correttamente ad una domanda di scienze, ma perde due punti se sbaglia
			una domanda di scienze.
	\item Geografo: guadagna due punti se risponde correttamente ad una domanda di geografia, ma perde due punti se sbaglia
			una domanda di geografia.
	\item Storico: guadagna due punti se risponde correttamente ad una domanda di storia, ma perde due punti se sbaglia
			una domanda di storia.
	\item Atleta: guadagna due punti se risponde correttamente ad una domanda di sport, ma perde due punti se sbaglia
			una domanda di sport.	
	\item Attore: guadagna due punti se risponde correttamente ad una domanda di spettacolo, ma perde due punti se sbaglia
			una domanda di spettacolo.
	\item Artista: guadagna due punti se risponde correttamente ad una domanda di arte, ma perde due punti se sbaglia
			una domanda di arte.
\end{itemize}


\chapter{Strutture dati utilizzate}
Segue successivamente un breve descrizione delle strutture dati utilizzate.
Si fa presente che non sono stati sfruttati set e tuple poiché non ritenuti necessari alle operazioni
da svolgere all'interno di questo progetto.

\section{Dizionari}
Innanzitutto sono stati utilizzati i dizionari, un tipo built-in, mutabile e non ordinato che contiene
elementi formati da una chiave (key) e un valore (value). Una volta che il dizionario è creato viene
popolato con un insieme di coppie (chiave:valore), e si può usare la chiave (che deve essere univoca) per
ottenere il valore corrispondente. E' proprio per queste caratteristiche che si è scelto di utilizzare
dei dizionari per implementare la memorizzazione degli utenti che accedono al server e associare il ruolo alla
materia corrispondente.

\section{Liste}
Una lista è una collezione mutabile di elementi indicizzata. Abbiamo usato questa collezione per memorizzare
le domande del gioco, per mantenere l'insieme dei giocatori, ed una per tenere traccia dei possibili ruoli da
assegnare ai giocatori stessi.
La lista contenente le domande viene popolata utilizzando un metodo della libreria json per leggere dal file
"questions.json" le domande stesse, la seconda lista serve per tenere traccia delle informazioni di ogni giocatore
(nome, punteggio, ruolo), mentre la terza è utile al momento dell'assegnamento dei ruoli: quando un nuovo player
entra nel gioco viene estratto un elemento casuale dalla questa lista.


\chapter{Thread attivi}

\section{Server}
Nel server vengono gestiti due thread: il primo si occupa di accettare le connessioni in entrata, 
che grazie al ciclo while rimangono in attesa occupandosi di immagazzinare tutti i client arrivati;
il secondo, invece, ha il compito di gestire i client e tutta la logica di gioco. Il suo unico parametro
è il client che deve gestire in modo tale da poter ottenere i messaggi che invia.
A seconda del messaggio che l’utente invierà verrà eseguita una diversa azione. Per far sì che il server
non si chiuda subito dopo essere partito, ovvero dopo l’accettazione delle connessioni, si utilizza il metodo
join che aspetta il completamento della funzione senza passare alla riga successiva.
Facendo diversamente causeremmo la chiusura del server.

\section{Client}
Per quanto riguarda il client viene utilizzato un unico thread che si occupa di gestire la ricezione
dei messaggi andando, poi ad aggiungerli alla vista che si occupa di mostrarli, realizzata tramite
la libreria tkinter.

\subsection{Timer}
All'inizio del gioco ogni client inizializza un timer, realizzato come un thread separato:
ogni secondo questo thread modifica una label (componente della libreria grafica tkinter)
apposita che tiene traccia del restante tempo di gioco.
Alla fine del conto alla rovescia, il thread termina e la label da esso modificata, così come una
parte della GUI destinata alla visualizzazione di domande e risposte, scompaiono dalla schermata di gioco.


\chapter{Struttura del progetto}
E' stato scelto di strutturare in parte il progetto ”ad oggetti” per integrare le conoscenze
acquisite nel corso ”Programmazione ad Oggetti” e sfruttare al massimo le potenzialità del linguaggio.

\section{RoadMap}
\begin{itemize}
	\item La classe Timer implementa il concetto di un timer che restituisce, attraverso il metodo "convert",
			un tempo in formato minuti:secondi.
	\item La classe Player non presenta metodi in sé per sé ma solo dei campi per memorizzare le informazioni
			del giocatore (nome, ruolo e punteggio).
	\item La classe Question, similmente alla classe Player, non presenta metodi ma solo dei campi per memorizzare
			le informazioni relative alla domanda (domanda, materia e risposta).
\end{itemize}


\chapter{Istruzioni per l'esecuzione}

\section{Eseguire il server}
Spostarsi all’interno della cartella ”src” ed eseguire il comando python3 server.py. Il server si metterà in ascolto sull'indirizzo IP ”127.0.0.1” e sulla porta ”53000”. Per modificare questi valori occorre modificare direttamente il file.

\section{Eseguire un client}
Spostarsi all’interno della cartella src ed eseguire il comando python3 client.py Il client proverà automaticamente a connettersi all’indirizzo IP ”127.0.0.1” e alla porta ”53000”. Per modificare questi valori occorre modificare
direttamente il file.

\end{document}